%************************************************
\chapter{Introduction}\label{ch:introduction}
%************************************************
The layout of the thesis is introduced here, from the basic scientific
background to the contents of each chapter, without repeating the
bibliographic references that are presented later in the text.
X-ray transmission imaging has been an extremely important tool for medical
diagnosis for over one hundred years: mobile and permanent radiographic
equiment were common in the hospitals around the battlefields of the first
world war. In France, Marie Curie organized the national effort to
provide emergency diagnostics through mobile X-ray machines operated ---
often by her personally --- with the current generated by a car engine. As a
result, an enormous number of wounded soldiers could be treated and saved.
In her own inspiring words, it was the first revolution of medicine through
imaging
\begin{quote}
    ``The total number of X-ray examinations during these two years [1917 and
    1918] can be evaluated as \num{1100000}. [\ldots]
    The history of radiology during the war offers a striking example of the
    unexpected amplitude that purely scientific discoveries can reach in
    certain conditions.

    The X-rays, whose marvelous properties have been applied to the
    examination and therapy of the human body almost immediately after their
    discovery, only found limited application until the outbreak of the war.
    [\ldots] Since then, what had seemed difficult and problematic became
    easy and received immediate solution; material and personnel
    multiplied as if by enchantment; those who did not understand gave in or
    accepted; those who did not know learned; those who were indifferent
    became devoted. Within a few years, a regulatory system was built, where
    doctors and surgeons conceive the possibility of ignoring the employment of
    X-rays as little as they had conceived their widespread usage
    before.``~\parencite{Curie1921}
\end{quote}

Exactly one century later, the properties of X-rays and their usefulness in
the field of medicine and beyond have indeed proven to be marvelous. 
New powerful techniques have been discovered, such as \ac{CT}, providing an
image reconstruction of an entire three-dimensional volume, with a
relatively cheap and high-resolution system.

More recently, synchrotron sources of X-rays opened up a whole new area of
research, relying on extremely high flux and coherence of the beam. The
possibility of creating X-ray beams with a high degree of spatial coherence
means that an entirely new method of generating contrast in the images is
availably: traditional absorption imaging produces an image based on how
many photons are absorbed in the sample compared to how many are generated
by the source; \emph{phase-contrast imaging} relies on detecting distortions
of a wave front impinging on an object. These distortions can only
be detected if the phases of the wave front at different locations can be
related to one another, hence the requirement for spatial coherence.
However, these phase differences in the amplitude of the fields describing
the radiation cannot be directly measured by a detector, since it reveals
only its squared modulus, the intensity. Several interferometric techniques
have been developed throughout the decades. They aim at translating these
phase differences in intensity modulations that can be recorded by a
detector. A review of phase-sensitive techniques, with their theorical
background, is presented in
chapter~\ref{ch:review}.

Talbot interferometry is one of these techniques. It relies on a physical
phenomenon discovered in 1836 for visible light, where a periodic structure
in a coherent beam of light creates an exact image of itself at a
macroscopic distance downstream. It was shown that this can be applied to
X-ray synchrotron and also to low brilliance, incoherent sources, after the
introduction of additional optical elements providing a minimum amount of
coherence, as summarized in section~\ref{sec:talbot-interferometry}.
A sample introduces distortions in the intensity modulation pattern
resulting from the self-image of the Talbot effect, and these distortions
can be quantitatively linked to physical properties of the sample itself
through the refractive index $n = 1 - \delta + i\beta$, where $\delta$ is
the term contributing to phase contrast and $\beta$ is the term responsible
for the traditional absorption imaging.

The scientific goals of this work are the extension of the applicability of
this technique to X-ray beams with higher energies than used so far.
Applications to mammography and lung imaging have been established, but
the imaging of sections or even a full the human body, with a greater thickness
and including bones with a high density, requires a larger penetration
power, which is achieved by increasing the acceleration voltage of the X-ray
tube. A larger penetration power is however an issue for the fabrication of
the optical elements for Talbot interferometry, which also need to partially block the
radiation. The fabrication of such elements for the
absorption of photons with an energy up to \SI{160}{\kilo\eV} is at present
not possible on an area of several tens of squared centimeters. Therefore,
an alternative arrangement of these structures has been realized, demonstrating
the possibility of imaging through phase contrast on X-ray sources with an
acceleration voltage between \SI{100}{\kilo\voltpeak} and
\SI{160}{\kilo\voltpeak}, as required by general purpose medical
examinations and material science scans involving materials with a large
atomic number. The realization and performance of two such systems is the
topic of chapter~\ref{ch:edgeon}.

After these first imaging results, a journey began towards
 the realization of larger systems with improved detectors, 
and a quantitative analysis of the recorded images. Previous research
had shown that it should be possible to link those to specific physical
properties of the investigated samples, even in the new energy range. A first step
in this direction is presented in chapter~\ref{ch:towards}.

A compromise between the technical possibilities of the
fabrication of the optical elements for the interferometers and the physical
requirements for high-energy interferometers has been reached.
This lead to the installation of a system suitable for the imaging of a two-dimensional
area with a high-efficiency detector, used for a detailed
study of the alveolar structure of lungs. Talbot interferometry is sensitive to
small microstructures in a sample, of the order of micrometers. A
quantitative model linking the exact microstructure as reconstructed from a
high resolution \ac{CT} scan to the values measured by an interferometric
radiography on our setup is described in chapter~\ref{ch:lung-dark-field}.
This model quantitatively predicts the resulting dark-field signal on an
interferometric radiography from a model of the lung as a collection of
spheres, without fits or free parameters. A predictive model opens up
possible applications to the diagnosis of pulmonary diseases involving
changes in the alveolar structure of the lungs.

Finally, chapter~\ref{ch:omnidirectional} is an attempt to go beyond
conventional Talbot interferometry by using an alternative design of the
periodic structures whose self-image is used for the generation of the
interference fringes. A Talbot-Lau grating consists of an array of lines that block the
incident X-ray radiation. This means that the sensitivity to the refraction
of X-rays is limited to the
direction orthogonal to
the grating lines. The installation of gratings with a novel structure
designed by our research group is
reported, allowing the retrieval of \emph{omnidirectional} data --- that is
data sensitive to microscopic deviations of the X-ray beam in all directions
--- with a single radiography.
