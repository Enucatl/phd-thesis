%*****************************************
\chapter{Review of Talbot-Lau interferometry}\label{ch:review}

\section{X-ray interaction with matter}

Immediately after their discovery in 1895, X-rays drew attention from the
medical community for their unprecedented ability of generating an image of
hidden part of a sample. What makes this kind of electromagnetic radiation
particularly useful as a probe is the fact that it interacts with matter
just strongly enough to carry information about the materials it passes
through, yet weakly enough to penetrate the entire volume.

In order to reconstruct some physical properties of a sample through the
observation of its interaction with X-rays, it is possible to introduce a
sequence of approximation to Maxwell's equations in a medium:
\begin{align}
    \nabla \cdot \vec{D} & = \rho \label{eq:gauss}\\
    \nabla \cdot \vec{B} & = 0 \label{eq:nomonopoles}\\
    \nabla \times \vec{E} + \partial_t \vec{B} & = 0
    \label{eq:induction}\\
    \nabla \times \vec{H} - \partial_t \vec{D} & = \vec{J}
    \label{eq:ampere}.
\end{align}

In general, the electric field \vec{E}, the magnetic field \vec{H}, the
electric displacement field \vec{D} and the magnetic induction field
\vec{B}, the charge density $\rho$ and the current density \vec{J} are
functions of the three spatial coordinates and time. 

However, a number of restrictions and approximations can be introduced to
simplify the equations. We are indeed interested in the case where there are
no charge or current sources

\begin{equation}
    \rho = 0\\
    \vec{J} = 0.
    \label{eq:absentsources}
\end{equation}

It follows that the electric and magnetic fields can be decoupled, since
taking the curl of~\eqref{eq:induction} yields

\begin{equation}
    \nabla \times (\nabla \times \vec{E}) + \partial_t(\nabla \cdot \vec{B}) = 0,
    \label{<++>}
\end{equation}
where the second summand is zero by equation~\eqref{eq:nomonopoles}:

\begin{align}
    \nabla \times (\nabla \times \vec{E}) &= 0\\
    \nabla^2 \vec{E} - \nabla (\nabla \cdot \vec{E}) = 0.
    \label{<++>}
\end{align}

Similarly, the electric field can be eliminated from
equation~\eqref{eq:ampere} by taking the curl and substituting
equation~\eqref{eq:gauss}.

We further restrict the materials to be isotropic and linear, that is
$\vec{D}(t, \vec{x}) = \varepsilon(\vec{x}) \vec{E}(t, \vec{x})$, where the
electric permeability coefficient is also independent of time. Moreover, we
exclude magnetic materials by setting $\vec{B} = \mu_0 \vec{H}$.

\begin{align*}
    \left(\varepsilon \mu_0 \partial^2_t - \nabla^2 \right)
    \vec{E} &= - \nabla(\nabla \cdot \vec{E}) \\
    \left(\varepsilon \mu_0 \partial^2_t - \nabla^2 \right)
    \vec{H} &= \frac{1}{\varepsilon}\nabla \varepsilon \times (\nabla \times
    \vec{H}).\\
\end{align*}

These two equations do not couple the magnetic and the electric field, but
they still mix different cartesian components of each components of each
field. If the electron density varies slowly with respect to the wavelength
of the X-rays, we can neglect the mixed derivative terms, and obtain a
scalar equation for each cartesian component of the electric field

\begin{equation}
    \left( \varepsilon(\vec{x}) \mu_0 \frac{\partial^2}{\partial t^2} - \nabla^2
    \right) \hat{\psi}(t, \vec{x}) = 0.\label{eq:helmoltz.spacetime}
\end{equation}

This equation, known as the Helmoltz equation is better understood when
taking the Fourier transform with respect to time:

\begin{equation*}
    \hat{\psi}(t, \vec{x}) =
    \frac{1}{\sqrt{2\pi}}\int_{0}^{\infty}\psi(\omega, \vec{x})
    e^{-i \omega t} \de{\omega}.
\end{equation*}

The time derivative becomes then $\partial_t = - i\omega = -i c k$, and the
equation can be finally written as

\begin{equation*}
    \left( \varepsilon_\omega(\vec{x}) \mu_0 c^2 k^2 + \nabla^2
    \right) \psi(\omega, \vec{x}) = 0.
\end{equation*}

Or, introducing the refractive index $n_\omega =
c\sqrt{\varepsilon_\omega\mu_0}$:

\begin{equation}
    \left( n_\omega^2(\vec{x}) k^2 + \nabla^2
    \right) \psi(\omega, \vec{x}) = 0.\label{eq:helmoltz.fourier}
\end{equation}

The fact that the interaction of X-rays with matter is both strong enough to
be a meaningful probe of physical properties and weak enough to penetrate an
entire macroscopical can be precisely expressed by the following
\emph{ansatz} to the solution of equation~\eqref{eq:helmoltz.fourier}:
    \begin{equation*}
 \psi(\omega, x, y, z) = \tilde{\psi}(\omega, x, y, z) \exp(ikz),    
\end{equation*}
that is, the solution is a product of a freely propagating plane wave
$\exp(ikz)$ and of a perturbation $\tilde{\psi}$ introduced by the sample.

With  $\nabla_\perp^2 = \partial_x^2 + \partial_y^2$, the Helmoltz
equation~\eqref{eq:helmoltz.fourier} becomes
\begin{equation*}
    \left[ 2ik \frac{\partial}{\partial z} + \nabla_\perp^2 +
    \frac{\partial^2}{\partial z ^2} + k^2 (n^2_\omega(\vec{x}) - 1)
\right]\tilde{\psi}(\omega, \vec{x}) = 0,
\end{equation*}
where the \emph{weak} interactions allow us to introduce the paraxial
approximation, where the variations along the $z$ axis, expressed by the
second derivative $\partial_z^2$ are negligible with
respect to the perpendicular $x-y$ plane. Moreover, the terms with
$\nabla_\perp^2$ can also be neglected as they are responsible of coupling
neighbouring X-ray trajectories.

Finally we can write the equation for the amplitude of the perturbation introduced by a
sample in the X-ray beam as
\begin{equation}
    \partial_z \tilde{\psi}(\omega, x, y, z) =
    \frac{k}{2i}(1 - n^2_\omega(\vec{x}))\tilde{\psi}(\omega, x, y, z).
    \label{eq:helmoltz.perturbation}
\end{equation}

The complex number $n_\omega$ is the refractive index, it is usually
very close to 1 in this energy range, and the imaginary and real parts are
usually explicitly considered:

\begin{equation*}
    n_\omega = 1 - \delta_\omega + i\beta_\omega.
\end{equation*}

Since both $\delta \ll 1$ and $\beta \ll 1$ the terms of order $\beta^2$ and
$\delta^$ can be neglected in equation~\eqref{eq:helmoltz.perturbation}.
The solution is then easily found by integrating with respect to $z$

\begin{equation}
    \tilde{\psi}(\omega, x, y, z_0) = \exp\left\{-ik\int_{z=0}^{z=z_0}(\delta_\omega - i
\beta_\omega) \de z \right\} \tilde{\psi}(\omega, x, y, 0),
\label{eq:helmoltz.solution}
\end{equation}
where the effects of the sample perturbations can be described as the sum of
a phase shift determined by the real part of the refractive index $\delta$
\begin{equation*}
    \Delta\varphi(x, y) = -k \int \delta_\omega(x, y, z) \de z
    %\label{eq:phase_deviation}
\end{equation*}
and an attenuation of the intensity, that is the square modulus of the
amplitude $I = |\psi|^2$, according to the Beer-Lambert law
\begin{equation}
    I(\omega, x, y, z_0) = \exp\left\{-2k\int \beta_\omega(x, y, z) \de
    z\right\}
    I(\omega, x, y, 0).\label{eq:beer-lambert}
\end{equation}

The coefficients $\delta$ and $\beta$ can be directly linked to the
interaction of X-rays with the electron clouds of the atoms that constitute
the sample

\begin{equation}
    n = 1 - \delta + i\beta = 1 - \frac{r_e\lambda^2}{2\pi}(f_1 + if_2).
    \label{eq:atom.factors}
\end{equation}

For all comparisons between measured and expected theoretical values in this
thesis, the
tabulated values of $f_1$ and $f_2$ from [NIST citation] together with
equation~\eqref{eq:atom.factors} have been used.

Far from absorption edges, these coefficients depend on the atomic number
$Z$ roughly as $\beta \propto Z^3 / k^4$ and $\delta \propto Z / k^2$,
which leads to the conclusion that the phase term $\delta$ becomes
relatively more important as the energy of the beam is increased. This is
the fundamental motivation of this experimental work, to assess the
feasibility and characteristics of X-ray interferometric techniques for
standard laboratory sources with a higher beam energy.

\section{Interferometry}

The results of the previous sections, and in particular,
equation~\eqref{eq:helmoltz.solution} point to the possibility of recovering
additional information on a sample through the knowledge of the real part of
the refractive index $\delta$. Unfortunately, only the intensity $I =
|\psi|^2$ of the X-ray beam can be directly detected, and an interferometric
setup is needed to convert phase differences into intensity modulations.
Various approaches have been presented in the last decades, on different
applications and source types.

\subsection{Propagation-based phase contrast}
Techniques based on free-space propagation of X-rays have been widely
explored on both monochromatic and polychromatic sources. If a weakly
absorbing object si placed at a distance $R_1$ from a coherent X-ray source,
it will introduce phase differences whose laplacian can be observed as an
intensity modulation at a distance $R_2$ downstream:
\begin{equation}
    I(x, y) = \frac{I_0}{M^2}\left(1 + \frac{R_2\lambda}{2\pi M^2}\nabla_\perp^2
    \varphi(x, y)\right).
    \label{eq:free.space.propagation}
\end{equation}
In general, this technique requires recording such diffraction patterns at
multiple distances $R_2$ from the sample in order to unambiguously
reconstruct the phase and therefore the electron density of the sample.
Moreover, a high spatial coherence of the beam is required, which means
that the source must be very small or placed at a large distance $R_1$ from
the sample. The detector also needs to have a high spatial resolution in
order to resolve the intensity modulations of
equation~\eqref{eq:free.space.propagation}. All of these constraints tend to
have a negative impact on exposure times, but a number of applications are
possible, including tomographic imaging and even \emph{in vivo} imaging.

\subsection{Diffraction enhanced imaging}
Diffraction enhanced --- or crystal-based --- methods employ two crystals:
one after the source, to obtain a monochromatic and collimated X-ray beam;
one before the detector, acting as an extremely sensitive angular filter.

The intensity on the detector is modulated by changing the incidence angle
$\vartheta$ of the beam on the second grating. The intensity in each pixel
is then recorded through multiple exposures as a function of $\vartheta$,
producing a \emph{rocking curve}. These rocking curves are sensitive to
refraction in a direction perpendicular to the beam and to the axis of
rotation of the analyzer grating, and can provide a differential phase
image, as well as a scattering contrast. This technique is also suitable for
polychromatic sources, although the presence of the first crystal acting as
a monochromator with low efficiency also leads to long exposure times,
especially on low brilliance laboratory sources. The beam is also collimated
to a fan beam on the plane and the sample needs to be scanned through it, in
addition to the scanning of the rocking curve. An device composed by an
array of crystals has been proposed as a solution to this issue.

\subsection{Bonse-Hart interferometry}
This method is the oldest approach to X-ray interferometry, where three
silicon beam splitters are used to create two paths from the X-ray source, a
sample is introduced in one of the paths, a second mirror deflects them onto
the third crystal, where they merge again forming an interference pattern
whose characteristics depend on the phase displacements introduced by the
sample. This third crystal is necessary since the fringes of this pattern
have the same period as Bragg planes and are therefore too small to be
directly recorded.

Again, as for diffraction enhanced techniques, the use of crystals as
optical elements means that only a very narrow bandwidth can be used from a
polychromatic source. Moreover, the three beam splitters should be
manufactured from a single crystal in order to reduce instabilities,
resulting in a limitation on the size of the field of view.

\subsection{Edge illumination}
Edge illumination is not an interferometric technique, although it is
inspired by diffraction enhanced imaging. The basic principle of edge
illumination is the collimation of the X-rays through a narrow aperture,
the resulting beam is then masked by a second aperture whose position is
scanned in a direction perpendicular to the beam: the intensity decreases from a
maximum when the two slits are perfectly aligned as the lateral displacement
of the second mask is increased.
The presence of a sample in the beam introduces refraction, which can be
observed as a displacement of this curve. The same applies to each pixel in
the detector, therefore the period of the source and detector masks have to
be matched so that there is a one-to-one mapping from each aperture in the
source mask to each aperture in the detector mask to each underlying
detector pixel.

This approach does not impose requirements on the spatial or temporal
coherence of the source, and has been successfully run on conventional
laboratory sources.

\subsection{Talbot interferometry}
Talbot interferometry in different configurations is the topic of this
thesis. The Talbot effect will be therefore described in detail.

The Helmoltz equation~\eqref{eq:helmoltz.fourier} provides the solution for
the propagation of a plane wave in a vacuum, where $n \equiv 1$. Plane waves
are solutions $\psi(\omega, \vec{x}) = e^{i\vec{k}\cdot \vec{x}}$ with $k^2
= k_x^2 + k_y^2 + k_z^2 = c^2 \omega^2$. We can then isolate the direction
of propagation $z$, by writing the solution in the form
\begin{equation}
    \psi(\omega, x, y, z) = e^{i(k_x x + k_y y)}e^{iz\sqrt{k^2 - k_x^2 -
    k_y^2}}.
    \label{eq:free.propagation}
\end{equation}
It is then clear that to propagate the plane wave from $z = 0$ to $z = z_0$
it is necessary to multiply by the factor $\exp(iz\sqrt{k^2 - k_x^2 -
k_y^2})$.

Given a general solution to the Helmoltz
equation~\eqref{eq:helmoltz.fourier}, it can be decomposed as a sum of plane
waves through the Fourier transform $\Fxy$ with respect to $x$ and $y$. Each
plane wave then propagates according to
equation~\eqref{eq:free.propagation}. The propagated wave can be then
recovered with the inverse transformation
\begin{equation}
    \psi(\omega, x, y, z_0) = \Fxy^{-1}e^{i z_0 \sqrt{k^2 - k_x^2 -
    k_y^2}}\Fxy    \psi(\omega, x, y, 0) 
    \label{eq:full.propagator}
\end{equation}
The paraxial approximation in this context translates to the fact that the
transverse components of the wave vector $\vec{k}$ are much smaller than the
component along the beam path $|k_x|, |k_y| \ll |k_z|$. The square root can
then be approximated to the first order $\sqrt{k^2 - k_x^2 - k_y^2} \approx
k - \frac{k_x^2 + k_y^2}{2k}$ to obtain the final form of the propagator
\begin{equation}
    \psi(\omega, x, y, z_0) = e^{i k z_0}\Fxy^{-1}e^{-i z_0\frac{k_x^2 +
    k_y^2}{2k}}\Fxy\psi(\omega, x, y, 0). \label{eq:paraxial.propagator}
\end{equation}

The optical phenomenon discovered in 1836 by Talbot is the observation that
the propagator~\eqref{eq:paraxial.propagator} is equal to one for a periodic
source wave at fixed distances, therefore called \emph{Talbot distances}.

Let's consider for simplicity the one-dimensional case where the
monochromatic field $\psi(\omega, x, z=0)$ is periodic with period $p_1$
along the $x$ direction. 

Then, following the recipe of equation~\eqref{eq:full.propagator}, we need
to calculate the Fourier transform with respect to $x$
\begin{equation*}
    \Fx\psi(\omega, x, 0) = \sum_j \psi_j(\omega)\delta(k_x -
    k_{xj}),
\end{equation*}
in terms of $k_{xj} = 2\pi j/ p_1$ and of the Dirac $\delta$.

By then propagating according to equation~\eqref{eq:paraxial.propagator}

\begin{align}
    \Fx\psi(\omega, x, z_0) &= e^{ikz_0} e^{-i z_0\frac{k_x^2}{2k}}\sum_j \psi_j(\omega)\delta(k_x -
    k_{xj}) \\
    &= e^{ikz_0}e^{-i z_0\frac{k_{xj}^2}{2k}}\sum_j \psi_j(\omega)\delta(k_x -
    k_{xj}).
    \label{eq:talbot1}
\end{align}

The effect of the Dirac $\delta$ is to replace $k_x$ with $k_{xj}$, which
means that the propagator will, apart from an irrelevant constant phase
factor, be equal to one at distances

\begin{equation*}
    \Delta_n = n \frac{p_1^2}{2 \lambda} \qquad n \in \mathbb{N}.
\end{equation*}

The source field at $z = 0$ creates a
self-image, exactly replicating its periodic intensity pattern at each
distance $\Delta_n$ downstream.

This effect can be observed by introducing a grating with absorbing lines in
the beam. However, the resulting intensity will be also reduced by a factor
known as the \emph{duty cycle} of the grating, that is the ratio of the
width of the opening to the pitch of the grating itself. This is not optimal
as it affects exposure times, but it is possible to observe the same effect
by introducing a phase periodicity in the source by letting the X-rays
go through a gratings whose lines change the phase of the radiation by a
factor of $\pi$. A self-image of the grating is then formed with double the
spatial frequency at the Lohmann distances
\begin{equation*}
    D_j = \left(j - \frac{1}{2}\right) \frac{p_1^2}{4 \lambda} \qquad
    j\in\mathbb{N}.
\end{equation*}

The intensity of a monochromatic periodic wave front propagating according to
equation~\eqref{eq:paraxial.propagator} is shown in
figure~\ref{fig:talbotcarpet}.

\begin{figure}[htb]
    \centering
    %% Creator: Matplotlib, PGF backend
%%
%% To include the figure in your LaTeX document, write
%%   \input{<filename>.pgf}
%%
%% Make sure the required packages are loaded in your preamble
%%   \usepackage{pgf}
%%
%% Figures using additional raster images can only be included by \input if
%% they are in the same directory as the main LaTeX file. For loading figures
%% from other directories you can use the `import` package
%%   \usepackage{import}
%% and then include the figures with
%%   \import{<path to file>}{<filename>.pgf}
%%
%% Matplotlib used the following preamble
%%   \usepackage{fontspec}
%%
\begingroup%
\makeatletter%
\begin{pgfpicture}%
\pgfpathrectangle{\pgfpointorigin}{\pgfqpoint{4.600000in}{4.000000in}}%
\pgfusepath{use as bounding box}%
\begin{pgfscope}%
\pgfsetrectcap%
\pgfsetroundjoin%
\definecolor{currentfill}{rgb}{1.000000,1.000000,1.000000}%
\pgfsetfillcolor{currentfill}%
\pgfsetlinewidth{0.000000pt}%
\definecolor{currentstroke}{rgb}{1.000000,1.000000,1.000000}%
\pgfsetstrokecolor{currentstroke}%
\pgfsetdash{}{0pt}%
\pgfpathmoveto{\pgfqpoint{0.000000in}{0.000000in}}%
\pgfpathlineto{\pgfqpoint{4.600000in}{0.000000in}}%
\pgfpathlineto{\pgfqpoint{4.600000in}{4.000000in}}%
\pgfpathlineto{\pgfqpoint{0.000000in}{4.000000in}}%
\pgfpathclose%
\pgfusepath{fill}%
\end{pgfscope}%
\begin{pgfscope}%
\pgfsetrectcap%
\pgfsetroundjoin%
\definecolor{currentfill}{rgb}{1.000000,1.000000,1.000000}%
\pgfsetfillcolor{currentfill}%
\pgfsetlinewidth{0.000000pt}%
\definecolor{currentstroke}{rgb}{0.000000,0.000000,0.000000}%
\pgfsetstrokecolor{currentstroke}%
\pgfsetdash{}{0pt}%
\pgfpathmoveto{\pgfqpoint{0.630833in}{0.516250in}}%
\pgfpathlineto{\pgfqpoint{4.397569in}{0.516250in}}%
\pgfpathlineto{\pgfqpoint{4.397569in}{3.835000in}}%
\pgfpathlineto{\pgfqpoint{0.630833in}{3.835000in}}%
\pgfpathclose%
\pgfusepath{fill}%
\end{pgfscope}%
\begin{pgfscope}%
\pgfpathrectangle{\pgfqpoint{0.630833in}{0.516250in}}{\pgfqpoint{3.766736in}{3.318750in}} %
\pgfusepath{clip}%
\pgftext[at=\pgfqpoint{0.630833in}{0.516250in},left,bottom]{\pgfimage[interpolate=true,width=3.770000in,height=3.323333in]{talbotcarpet-img0.png}}%
\end{pgfscope}%
\begin{pgfscope}%
\pgfsetbuttcap%
\pgfsetroundjoin%
\definecolor{currentfill}{rgb}{0.000000,0.000000,0.000000}%
\pgfsetfillcolor{currentfill}%
\pgfsetlinewidth{1.003750pt}%
\definecolor{currentstroke}{rgb}{0.000000,0.000000,0.000000}%
\pgfsetstrokecolor{currentstroke}%
\pgfsetdash{}{0pt}%
\pgfsys@defobject{currentmarker}{\pgfqpoint{0.000000in}{0.000000in}}{\pgfqpoint{0.000000in}{0.055556in}}{%
\pgfpathmoveto{\pgfqpoint{0.000000in}{0.000000in}}%
\pgfpathlineto{\pgfqpoint{0.000000in}{0.055556in}}%
\pgfusepath{stroke,fill}%
}%
\begin{pgfscope}%
\pgfsys@transformshift{0.631461in}{0.516250in}%
\pgfsys@useobject{currentmarker}{}%
\end{pgfscope}%
\end{pgfscope}%
\begin{pgfscope}%
\pgftext[left,bottom,x=0.594030in,y=0.342000in,rotate=0.000000]{{\rmfamily\fontsize{11.000000}{13.200000}\selectfont 0}}
%
\end{pgfscope}%
\begin{pgfscope}%
\pgfsetbuttcap%
\pgfsetroundjoin%
\definecolor{currentfill}{rgb}{0.000000,0.000000,0.000000}%
\pgfsetfillcolor{currentfill}%
\pgfsetlinewidth{1.003750pt}%
\definecolor{currentstroke}{rgb}{0.000000,0.000000,0.000000}%
\pgfsetstrokecolor{currentstroke}%
\pgfsetdash{}{0pt}%
\pgfsys@defobject{currentmarker}{\pgfqpoint{0.000000in}{0.000000in}}{\pgfqpoint{0.000000in}{0.055556in}}{%
\pgfpathmoveto{\pgfqpoint{0.000000in}{0.000000in}}%
\pgfpathlineto{\pgfqpoint{0.000000in}{0.055556in}}%
\pgfusepath{stroke,fill}%
}%
\begin{pgfscope}%
\pgfsys@transformshift{1.384683in}{0.516250in}%
\pgfsys@useobject{currentmarker}{}%
\end{pgfscope}%
\end{pgfscope}%
\begin{pgfscope}%
\pgftext[left,bottom,x=1.347252in,y=0.345208in,rotate=0.000000]{{\rmfamily\fontsize{11.000000}{13.200000}\selectfont 1}}
%
\end{pgfscope}%
\begin{pgfscope}%
\pgfsetbuttcap%
\pgfsetroundjoin%
\definecolor{currentfill}{rgb}{0.000000,0.000000,0.000000}%
\pgfsetfillcolor{currentfill}%
\pgfsetlinewidth{1.003750pt}%
\definecolor{currentstroke}{rgb}{0.000000,0.000000,0.000000}%
\pgfsetstrokecolor{currentstroke}%
\pgfsetdash{}{0pt}%
\pgfsys@defobject{currentmarker}{\pgfqpoint{0.000000in}{0.000000in}}{\pgfqpoint{0.000000in}{0.055556in}}{%
\pgfpathmoveto{\pgfqpoint{0.000000in}{0.000000in}}%
\pgfpathlineto{\pgfqpoint{0.000000in}{0.055556in}}%
\pgfusepath{stroke,fill}%
}%
\begin{pgfscope}%
\pgfsys@transformshift{2.137904in}{0.516250in}%
\pgfsys@useobject{currentmarker}{}%
\end{pgfscope}%
\end{pgfscope}%
\begin{pgfscope}%
\pgftext[left,bottom,x=2.042265in,y=0.342000in,rotate=0.000000]{{\rmfamily\fontsize{11.000000}{13.200000}\selectfont 1.5}}
%
\end{pgfscope}%
\begin{pgfscope}%
\pgfsetbuttcap%
\pgfsetroundjoin%
\definecolor{currentfill}{rgb}{0.000000,0.000000,0.000000}%
\pgfsetfillcolor{currentfill}%
\pgfsetlinewidth{1.003750pt}%
\definecolor{currentstroke}{rgb}{0.000000,0.000000,0.000000}%
\pgfsetstrokecolor{currentstroke}%
\pgfsetdash{}{0pt}%
\pgfsys@defobject{currentmarker}{\pgfqpoint{0.000000in}{0.000000in}}{\pgfqpoint{0.000000in}{0.055556in}}{%
\pgfpathmoveto{\pgfqpoint{0.000000in}{0.000000in}}%
\pgfpathlineto{\pgfqpoint{0.000000in}{0.055556in}}%
\pgfusepath{stroke,fill}%
}%
\begin{pgfscope}%
\pgfsys@transformshift{2.891126in}{0.516250in}%
\pgfsys@useobject{currentmarker}{}%
\end{pgfscope}%
\end{pgfscope}%
\begin{pgfscope}%
\pgftext[left,bottom,x=2.853695in,y=0.345208in,rotate=0.000000]{{\rmfamily\fontsize{11.000000}{13.200000}\selectfont 2}}
%
\end{pgfscope}%
\begin{pgfscope}%
\pgfsetbuttcap%
\pgfsetroundjoin%
\definecolor{currentfill}{rgb}{0.000000,0.000000,0.000000}%
\pgfsetfillcolor{currentfill}%
\pgfsetlinewidth{1.003750pt}%
\definecolor{currentstroke}{rgb}{0.000000,0.000000,0.000000}%
\pgfsetstrokecolor{currentstroke}%
\pgfsetdash{}{0pt}%
\pgfsys@defobject{currentmarker}{\pgfqpoint{0.000000in}{0.000000in}}{\pgfqpoint{0.000000in}{0.055556in}}{%
\pgfpathmoveto{\pgfqpoint{0.000000in}{0.000000in}}%
\pgfpathlineto{\pgfqpoint{0.000000in}{0.055556in}}%
\pgfusepath{stroke,fill}%
}%
\begin{pgfscope}%
\pgfsys@transformshift{3.644348in}{0.516250in}%
\pgfsys@useobject{currentmarker}{}%
\end{pgfscope}%
\end{pgfscope}%
\begin{pgfscope}%
\pgftext[left,bottom,x=3.548709in,y=0.342000in,rotate=0.000000]{{\rmfamily\fontsize{11.000000}{13.200000}\selectfont 2.5}}
%
\end{pgfscope}%
\begin{pgfscope}%
\pgfsetbuttcap%
\pgfsetroundjoin%
\definecolor{currentfill}{rgb}{0.000000,0.000000,0.000000}%
\pgfsetfillcolor{currentfill}%
\pgfsetlinewidth{1.003750pt}%
\definecolor{currentstroke}{rgb}{0.000000,0.000000,0.000000}%
\pgfsetstrokecolor{currentstroke}%
\pgfsetdash{}{0pt}%
\pgfsys@defobject{currentmarker}{\pgfqpoint{0.000000in}{0.000000in}}{\pgfqpoint{0.000000in}{0.055556in}}{%
\pgfpathmoveto{\pgfqpoint{0.000000in}{0.000000in}}%
\pgfpathlineto{\pgfqpoint{0.000000in}{0.055556in}}%
\pgfusepath{stroke,fill}%
}%
\begin{pgfscope}%
\pgfsys@transformshift{4.397569in}{0.516250in}%
\pgfsys@useobject{currentmarker}{}%
\end{pgfscope}%
\end{pgfscope}%
\begin{pgfscope}%
\pgftext[left,bottom,x=4.360139in,y=0.342000in,rotate=0.000000]{{\rmfamily\fontsize{11.000000}{13.200000}\selectfont 3}}
%
\end{pgfscope}%
\begin{pgfscope}%
\pgftext[left,bottom,x=1.827083in,y=0.165000in,rotate=0.000000]{{\rmfamily\fontsize{11.000000}{13.200000}\selectfont distanze di Lohmann}}
%
\end{pgfscope}%
\begin{pgfscope}%
\pgfsetbuttcap%
\pgfsetroundjoin%
\definecolor{currentfill}{rgb}{0.000000,0.000000,0.000000}%
\pgfsetfillcolor{currentfill}%
\pgfsetlinewidth{1.003750pt}%
\definecolor{currentstroke}{rgb}{0.000000,0.000000,0.000000}%
\pgfsetstrokecolor{currentstroke}%
\pgfsetdash{}{0pt}%
\pgfsys@defobject{currentmarker}{\pgfqpoint{0.000000in}{0.000000in}}{\pgfqpoint{0.055556in}{0.000000in}}{%
\pgfpathmoveto{\pgfqpoint{0.000000in}{0.000000in}}%
\pgfpathlineto{\pgfqpoint{0.055556in}{0.000000in}}%
\pgfusepath{stroke,fill}%
}%
\begin{pgfscope}%
\pgfsys@transformshift{0.630833in}{0.517633in}%
\pgfsys@useobject{currentmarker}{}%
\end{pgfscope}%
\end{pgfscope}%
\begin{pgfscope}%
\pgftext[left,bottom,x=0.370111in,y=0.465230in,rotate=0.000000]{{\rmfamily\fontsize{11.000000}{13.200000}\selectfont 0.0}}
%
\end{pgfscope}%
\begin{pgfscope}%
\pgfsetbuttcap%
\pgfsetroundjoin%
\definecolor{currentfill}{rgb}{0.000000,0.000000,0.000000}%
\pgfsetfillcolor{currentfill}%
\pgfsetlinewidth{1.003750pt}%
\definecolor{currentstroke}{rgb}{0.000000,0.000000,0.000000}%
\pgfsetstrokecolor{currentstroke}%
\pgfsetdash{}{0pt}%
\pgfsys@defobject{currentmarker}{\pgfqpoint{0.000000in}{0.000000in}}{\pgfqpoint{0.055556in}{0.000000in}}{%
\pgfpathmoveto{\pgfqpoint{0.000000in}{0.000000in}}%
\pgfpathlineto{\pgfqpoint{0.055556in}{0.000000in}}%
\pgfusepath{stroke,fill}%
}%
\begin{pgfscope}%
\pgfsys@transformshift{0.630833in}{1.070758in}%
\pgfsys@useobject{currentmarker}{}%
\end{pgfscope}%
\end{pgfscope}%
\begin{pgfscope}%
\pgftext[left,bottom,x=0.370111in,y=1.018355in,rotate=0.000000]{{\rmfamily\fontsize{11.000000}{13.200000}\selectfont 0.5}}
%
\end{pgfscope}%
\begin{pgfscope}%
\pgfsetbuttcap%
\pgfsetroundjoin%
\definecolor{currentfill}{rgb}{0.000000,0.000000,0.000000}%
\pgfsetfillcolor{currentfill}%
\pgfsetlinewidth{1.003750pt}%
\definecolor{currentstroke}{rgb}{0.000000,0.000000,0.000000}%
\pgfsetstrokecolor{currentstroke}%
\pgfsetdash{}{0pt}%
\pgfsys@defobject{currentmarker}{\pgfqpoint{0.000000in}{0.000000in}}{\pgfqpoint{0.055556in}{0.000000in}}{%
\pgfpathmoveto{\pgfqpoint{0.000000in}{0.000000in}}%
\pgfpathlineto{\pgfqpoint{0.055556in}{0.000000in}}%
\pgfusepath{stroke,fill}%
}%
\begin{pgfscope}%
\pgfsys@transformshift{0.630833in}{1.623883in}%
\pgfsys@useobject{currentmarker}{}%
\end{pgfscope}%
\end{pgfscope}%
\begin{pgfscope}%
\pgftext[left,bottom,x=0.370111in,y=1.571480in,rotate=0.000000]{{\rmfamily\fontsize{11.000000}{13.200000}\selectfont 1.0}}
%
\end{pgfscope}%
\begin{pgfscope}%
\pgfsetbuttcap%
\pgfsetroundjoin%
\definecolor{currentfill}{rgb}{0.000000,0.000000,0.000000}%
\pgfsetfillcolor{currentfill}%
\pgfsetlinewidth{1.003750pt}%
\definecolor{currentstroke}{rgb}{0.000000,0.000000,0.000000}%
\pgfsetstrokecolor{currentstroke}%
\pgfsetdash{}{0pt}%
\pgfsys@defobject{currentmarker}{\pgfqpoint{0.000000in}{0.000000in}}{\pgfqpoint{0.055556in}{0.000000in}}{%
\pgfpathmoveto{\pgfqpoint{0.000000in}{0.000000in}}%
\pgfpathlineto{\pgfqpoint{0.055556in}{0.000000in}}%
\pgfusepath{stroke,fill}%
}%
\begin{pgfscope}%
\pgfsys@transformshift{0.630833in}{2.177008in}%
\pgfsys@useobject{currentmarker}{}%
\end{pgfscope}%
\end{pgfscope}%
\begin{pgfscope}%
\pgftext[left,bottom,x=0.370111in,y=2.124605in,rotate=0.000000]{{\rmfamily\fontsize{11.000000}{13.200000}\selectfont 1.5}}
%
\end{pgfscope}%
\begin{pgfscope}%
\pgfsetbuttcap%
\pgfsetroundjoin%
\definecolor{currentfill}{rgb}{0.000000,0.000000,0.000000}%
\pgfsetfillcolor{currentfill}%
\pgfsetlinewidth{1.003750pt}%
\definecolor{currentstroke}{rgb}{0.000000,0.000000,0.000000}%
\pgfsetstrokecolor{currentstroke}%
\pgfsetdash{}{0pt}%
\pgfsys@defobject{currentmarker}{\pgfqpoint{0.000000in}{0.000000in}}{\pgfqpoint{0.055556in}{0.000000in}}{%
\pgfpathmoveto{\pgfqpoint{0.000000in}{0.000000in}}%
\pgfpathlineto{\pgfqpoint{0.055556in}{0.000000in}}%
\pgfusepath{stroke,fill}%
}%
\begin{pgfscope}%
\pgfsys@transformshift{0.630833in}{2.730133in}%
\pgfsys@useobject{currentmarker}{}%
\end{pgfscope}%
\end{pgfscope}%
\begin{pgfscope}%
\pgftext[left,bottom,x=0.370111in,y=2.677730in,rotate=0.000000]{{\rmfamily\fontsize{11.000000}{13.200000}\selectfont 2.0}}
%
\end{pgfscope}%
\begin{pgfscope}%
\pgfsetbuttcap%
\pgfsetroundjoin%
\definecolor{currentfill}{rgb}{0.000000,0.000000,0.000000}%
\pgfsetfillcolor{currentfill}%
\pgfsetlinewidth{1.003750pt}%
\definecolor{currentstroke}{rgb}{0.000000,0.000000,0.000000}%
\pgfsetstrokecolor{currentstroke}%
\pgfsetdash{}{0pt}%
\pgfsys@defobject{currentmarker}{\pgfqpoint{0.000000in}{0.000000in}}{\pgfqpoint{0.055556in}{0.000000in}}{%
\pgfpathmoveto{\pgfqpoint{0.000000in}{0.000000in}}%
\pgfpathlineto{\pgfqpoint{0.055556in}{0.000000in}}%
\pgfusepath{stroke,fill}%
}%
\begin{pgfscope}%
\pgfsys@transformshift{0.630833in}{3.283258in}%
\pgfsys@useobject{currentmarker}{}%
\end{pgfscope}%
\end{pgfscope}%
\begin{pgfscope}%
\pgftext[left,bottom,x=0.370111in,y=3.230855in,rotate=0.000000]{{\rmfamily\fontsize{11.000000}{13.200000}\selectfont 2.5}}
%
\end{pgfscope}%
\begin{pgfscope}%
\pgftext[left,bottom,x=0.300667in,y=1.742156in,rotate=90.000000]{{\rmfamily\fontsize{11.000000}{13.200000}\selectfont periodi di \(\displaystyle G_1\)}}
%
\end{pgfscope}%
\begin{pgfscope}%
\pgfsetrectcap%
\pgfsetroundjoin%
\pgfsetlinewidth{1.003750pt}%
\definecolor{currentstroke}{rgb}{0.000000,0.000000,0.000000}%
\pgfsetstrokecolor{currentstroke}%
\pgfsetdash{}{0pt}%
\pgfpathmoveto{\pgfqpoint{0.630833in}{3.835000in}}%
\pgfpathlineto{\pgfqpoint{4.397569in}{3.835000in}}%
\pgfusepath{stroke}%
\end{pgfscope}%
\begin{pgfscope}%
\pgfsetrectcap%
\pgfsetroundjoin%
\pgfsetlinewidth{1.003750pt}%
\definecolor{currentstroke}{rgb}{0.000000,0.000000,0.000000}%
\pgfsetstrokecolor{currentstroke}%
\pgfsetdash{}{0pt}%
\pgfpathmoveto{\pgfqpoint{4.397569in}{0.516250in}}%
\pgfpathlineto{\pgfqpoint{4.397569in}{3.835000in}}%
\pgfusepath{stroke}%
\end{pgfscope}%
\begin{pgfscope}%
\pgfsetrectcap%
\pgfsetroundjoin%
\pgfsetlinewidth{1.003750pt}%
\definecolor{currentstroke}{rgb}{0.000000,0.000000,0.000000}%
\pgfsetstrokecolor{currentstroke}%
\pgfsetdash{}{0pt}%
\pgfpathmoveto{\pgfqpoint{0.630833in}{0.516250in}}%
\pgfpathlineto{\pgfqpoint{4.397569in}{0.516250in}}%
\pgfusepath{stroke}%
\end{pgfscope}%
\begin{pgfscope}%
\pgfsetrectcap%
\pgfsetroundjoin%
\pgfsetlinewidth{1.003750pt}%
\definecolor{currentstroke}{rgb}{0.000000,0.000000,0.000000}%
\pgfsetstrokecolor{currentstroke}%
\pgfsetdash{}{0pt}%
\pgfpathmoveto{\pgfqpoint{0.630833in}{0.516250in}}%
\pgfpathlineto{\pgfqpoint{0.630833in}{3.835000in}}%
\pgfusepath{stroke}%
\end{pgfscope}%
\end{pgfpicture}%
\makeatother%
\endgroup%

    \caption[Tappeto di Talbot.]{Simulazione del \emph{tappeto di Talbot},
    ovvero della propagazione di un fronte
    d'onda con periodicit\`a di fase. Si osserva la formazione di frange luminose alle
    distanze di Lohmann con periodo dimezzato rispetto al reticolo \G{1}.}
    \label{fig:talbotcarpet}
\end{figure}

\section{Talbot-Lau interferometry}
\subsection{Image generation and analysis}
The fundamental principle of Talbot interferometry lies in the self-imaging
property of a periodic wave front as described in the previous section. A
sample can then be introduced in the beam, resulting in a reshaping of the
interference pattern.

Specifically, refraction of the X-ray beam produces a lateral displacement
of the interference fringes. The direction of propagation of a plane wave
\begin{equation}
    e^{-i \vec{k} \cdot \vec{x}} = e^{-i\varphi(x)}
    \label{eq:plane.wave}
\end{equation}
is determined by the unit vector
\begin{equation}
    \vec{n} = -\frac{1}{k} \nabla\varphi.
    \label{eq:plane.wave.direction}
\end{equation}
Therefore, the angular deviation of the beam along the axis $x$ perpendicular to
the direction of propagation and to the direction of the grating lines is
\begin{equation}
        \alpha = -\frac{1}{k} \partial_x\varphi.\label{eq:refraction.angle}
\end{equation}
This deviation would be measurable as a lateral displacement of the
interference fringes of the Talbot self-image of the grating \G1, with the
\emph{caveat} that this interference pattern has a pitch of the order of
few micrometers and is therefore not resolved by detectors with a pixel size
that is one or two orders of magnitude larger, although some applications
exist on synchrotron sources for a limited field of view\cn.

For this reason, a second grating \G2 with absorbing lines is placed at the
location where the self-image of \G1 is created with the same period of the
interference pattern. This grating is then laterally displaced in the $x$
direction: when the grating lines of \G2
match the bright lines of the interference pattern, the intensity recorded
by the underlying detector pixel reaches a minimum, whereas when the
absorbing lines are superimposed to the dark lines the radiation is able to
go through and a maximum intensity is recorded. Sliding the grating \G2
along the rectangular interference pattern of \G1 corresponds to recording
the convolution of the two signals. The result is a triangular signal, known
as the \emph{phase stepping curve}.

A triangular phase stepping curve is the result of the convolution of the
two rectangular shapes of the grating transmissions. However, a finite
source size also affects the shape of the curve by convolving the signal
with the image of the source. The result is calculated in detail in\cn, we
are here interested in the result pictured in
figure~\ref{fig:phase.stepping}: the curve is well approximated by a
sinusoid
\begin{equation*}
    s(x) = a_0 + a_1 \cos \left(\frac{2 \pi}{p_2} x + \theta\right).
\end{equation*}

A number of \emph{phase steps} are then recorded for different displacements
$x$ of \G2 and the three parameters $a_0$, $a_1$ and $\theta$ are determined
with a linear fit.

A reference, or \emph{flat}, curve is then compared to the curve observed
with the sample in the beam, or \emph{sample} curve

\begin{align}
    s_f(x) &= a_{0,f} + a_{1,f} \cos\left(\frac{2 \pi}{p_2} x + \theta_{f}\right)\\
    s_s(x) &= a_{0,s} + a_{1,s} \cos\left(\frac{2 \pi}{p_2} x +
    \theta_{s}\right).
    \label{eq:flat}
\end{align}

Three complementary signals can be determined by these parameters:
\begin{description}
    \item[transmission,] given by the ratio $A = a_{0,c} / a_{0,f}$, is the
        intensity of the transmitted radiation, as in conventional
        radiography, related to the Beer-Lambert
        law~\eqref{eq:beer-lambert}.
    \item[differential phase,] or the difference $P = \theta_{c} -
        \theta_f$, depending on the lateral displacement of the interference
        fringes given by
        $\alpha$~\eqref{eq:refraction.angle} at the
        $j$th Lohmann distance according to
        \begin{equation*}
            P = 2\pi \frac{D_j}{p_2}\alpha.
        \end{equation*}
        It is a differential phase signal because it is proportional to the
        derivative of the phase displacement introduced by the sample.
    \item[visibility reduction,] also known as 
        \emph{scattering} or \emph{dark field} contrast. The
        \emph{visibility} of the curve is the parameter
        $v = 2a_1 / a_0$. The ratio
        \begin{equation*}
            B = \frac{v_c}{v_f} =
            \frac{a_{1,c}}{a_{0,c}}\frac{a_{0,f}}{a_{1,f}}
        \end{equation*}
        is influenced by inhomogeneities of the sample on a scale smaller
        than a detector pixel\cn.
\end{description}

\subsection{Spatial and temporal coherence}
Talbot interferometry is compatible with laboratory sources with a wide
\emph{bremsstrahlung} spectrum. It has been shown\cn that the spectral
acceptance at the $j$th Lohmann distance is
\begin{equation}
    \frac{\Delta \lambda}{\lambda} = \frac{1}{2j - 1}.\label{eq:acceptance}
\end{equation}
Our experiments are designed for sources with a high voltage and a wide
spectrum. Therefore, given the inverse proportionality between Lohmann order
and spectral acceptance, our interferometers operate at the first order.

Spatial coherence is more critical for the formation of the self-image of
the grating \G1, in the direction perpendicular to the grating lines --- since no
interference is observed along the grating lines there
is no coherence requirement in that direction.

Let's consider two point sources producing an interference pattern at a
distance $D_j$ from \G1, if the separation between the two points is
$\epsilon$, the two patterns will be shifted by $\epsilon D_j / \ell$: if
this distance is such that the bright lines of one pattern
overlap with the dark lines of the second pattern, the interference will
disappear. We can therefore establish a maximum source width for the
interference effect to occur as
\begin{equation}
    s < \frac{p_2\ell}{2D_j}.
    \label{eq:source.size}
\end{equation}

Synchrotron sources can achieve the required coherence without additional
optical elements, but laboratory sources have a much larger focal spot. If
for instance $\ell = \SI{1}{\meter}$, $\D_j = \SI{10}{\centi\meter}$, then
the source has to be smaller than \SI{5}{\micro\meter} which is only
achievable with a microfocus source, which can only provide a very small
amount of radiation for imaging.

This problem can be overcome by placing an additional grating \G0 in front
of the source. This grating creates an array of individually coherent but
mutually incoherent sources, whose interference patterns superimpose with the
geometrical constraint
\begin{equation}
    p_0 = p_2 \frac{\ell}{D_j}.\label{eq:p0}
\end{equation}

Laboratory sources on a compact setup which are the goal of our experiments
also introduce a significant geometrical magnification. The equations above
all assume a plane wave geometry, but this can be easily adapted for a
spherical wave front according to~\cite{Engelhardt2008}.

The Lohmann distances are rescaled according to
\begin{equation*}
    D_j^\prime = D_j \frac{1}{1 - \dfrac{D_j}{\ell}}.
\end{equation*}

The period of the grating $p_2$ is rescaled by the same factor
\begin{equation}
    p_2^\prime = p_2 \frac{1}{1 -
        \dfrac{D_j}{\ell}}.\label{eq:magnification}
\end{equation}
