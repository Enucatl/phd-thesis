%*****************************************
\chapter{Review of Talbot-Lau interferometry}\label{ch:review}

\section{X-ray interaction with matter}

Immediately after their discovery in 1895, X-rays drew attention from the
medical community for their unprecedented ability of generating an image of
hidden part of a sample. What makes this kind of electromagnetic radiation
particularly useful as a probe is the fact that it interacts with matter
just strongly enough to carry information about the materials it passes
through, yet weakly enough to penetrate the entire volume.

In order to reconstruct some physical properties of a sample through the
observation of its interaction with X-rays, it is possible to introduce a
sequence of approximation to Maxwell's equations in a medium:
\begin{align}
    \nabla \cdot \vec{D} & = \rho \label{eq:gauss}\\
    \nabla \cdot \vec{B} & = 0 \label{eq:nomonopoles}\\
    \nabla \times \vec{E} + \partial_t \vec{B} & = 0
    \label{eq:induction}\\
    \nabla \times \vec{H} - \partial_t \vec{D} & = \vec{J}
    \label{eq:ampere}.
\end{align}

In general, the electric field \vec{E}, the magnetic field \vec{H}, the
electric displacement field \vec{D} and the magnetic induction field
\vec{B}, the charge density $\rho$ and the current density \vec{J} are
functions of the three spatial coordinates and time. 

However, a number of restrictions and approximations can be introduced to
simplify the equations. We are indeed interested in the case where there are
no charge or current sources

\begin{equation}
    \rho = 0\\
    \vec{J} = 0.
    \label{eq:absentsources}
\end{equation}

It follows that the electric and magnetic fields can be decoupled, since
taking the curl of~\eqref{eq:induction} yields

\begin{equation}
    \nabla \times (\nabla \times \vec{E}) + \partial_t(\nabla \cdot \vec{B}) = 0,
    \label{<++>}
\end{equation}
where the second summand is zero by equation~\eqref{eq:nomonopoles}:

\begin{align}
    \nabla \times (\nabla \times \vec{E}) &= 0\\
    \nabla^2 \vec{E} - \nabla (\nabla \cdot \vec{E}) = 0.
    \label{<++>}
\end{align}

Similarly, the electric field can be eliminated from
equation~\eqref{eq:ampere} by taking the curl and substituting
equation~\eqref{eq:gauss}.

We further restrict the materials to be isotropic and linear, that is
$\vec{D}(t, \vec{x}) = \varepsilon(\vec{x}) \vec{E}(t, \vec{x})$, where the
electric permeability coefficient is also independent of time. Moreover, we
exclude magnetic materials by setting $\vec{B} = \mu_0 \vec{H}$.

\begin{align*}
    \left(\varepsilon \mu_0 \partial^2_t - \nabla^2 \right)
    \vec{E} &= - \nabla(\nabla \cdot \vec{E}) \\
    \left(\varepsilon \mu_0 \partial^2_t - \nabla^2 \right)
    \vec{H} &= \frac{1}{\varepsilon}\nabla \varepsilon \times (\nabla \times
    \vec{H}).\\
\end{align*}

These two equations do not couple the magnetic and the electric field, but
they still mix different cartesian components of each components of each
field. If the electron density varies slowly with respect to the wavelength
of the X-rays, we can neglect the mixed derivative terms, and obtain a
scalar equation for each cartesian component of the electric field

\begin{equation}
    \left( \varepsilon(\vec{x}) \mu_0 \frac{\partial^2}{\partial t^2} - \nabla^2
    \right) \hat{\psi}(t, \vec{x}) = 0.\label{eq:helmoltz.spacetime}
\end{equation}

This equation, known as the Helmoltz equation is better understood when
taking the Fourier transform with respect to time:

\begin{equation*}
    \hat{\psi}(t, \vec{x}) =
    \frac{1}{\sqrt{2\pi}}\int_{0}^{\infty}\psi(\omega, \vec{x})
    e^{-i \omega t} \de{\omega}.
\end{equation*}

The time derivative becomes then $\partial_t = - i\omega = -i c k$, and the
equation can be finally written as

\begin{equation*}
    \left( \varepsilon_\omega(\vec{x}) \mu_0 c^2 k^2 + \nabla^2
    \right) \psi(\omega, \vec{x}) = 0.
\end{equation*}

Or, introducing the refractive index $n_\omega =
c\sqrt{\varepsilon_\omega\mu_0}$:

\begin{equation}
    \left( n_\omega^2(\vec{x}) k^2 + \nabla^2
    \right) \psi(\omega, \vec{x}) = 0.\label{eq:helmoltz.fourier}
\end{equation}

The fact that the interaction of X-rays with matter is both strong enough to
be a meaningful probe of physical properties and weak enough to penetrate an
entire macroscopical can be precisely expressed by the following
\emph{ansatz} to the solution of equation~\eqref{eq:helmoltz.fourier}:
    \begin{equation*}
 \psi(\omega, x, y, z) = \tilde{\psi}(\omega, x, y, z) \exp(ikz),    
\end{equation*}
that is, the solution is a product of a freely propagating plane wave
$\exp(ikz)$ and of a perturbation $\tilde{\psi}$ introduced by the sample.

With  $\nabla_\perp^2 = \partial_x^2 + \partial_y^2$, the Helmoltz
equation~\eqref{eq:helmoltz.fourier} becomes
\begin{equation*}
    \left[ 2ik \frac{\partial}{\partial z} + \nabla_\perp^2 +
    \frac{\partial^2}{\partial z ^2} + k^2 (n^2_\omega(\vec{x}) - 1)
\right]\tilde{\psi}(\omega, \vec{x}) = 0,
\end{equation*}
where the \emph{weak} interactions allow us to introduce the paraxial
approximation, where the variations along the $z$ axis, expressed by the
second derivative $\partial_z^2$ are negligible with
respect to the perpendicular $x-y$ plane. Moreover, the terms with
$\nabla_\perp^2$ can also be neglected as they are responsible of coupling
neighbouring X-ray trajectories.

Finally we can write the equation for the amplitude of the perturbation introduced by a
sample in the X-ray beam as
\begin{equation}
    \partial_z \tilde{\psi}(\omega, x, y, z) =
    \frac{k}{2i}(1 - n^2_\omega(\vec{x}))\tilde{\psi}(\omega, x, y, z).
    \label{eq:helmoltz.perturbation}
\end{equation}

The complex number $n_\omega$ is the refractive index, it is usually
very close to 1 in this energy range, and the imaginary and real parts are
usually explicitly considered:

\begin{equation*}
    n_\omega = 1 - \delta_\omega + i\beta_\omega.
\end{equation*}

Since both $\delta \ll 1$ and $\beta \ll 1$ the terms of order $\beta^2$ and
$\delta^$ can be neglected in equation~\eqref{eq:helmoltz.perturbation}.
The solution is then easily found by integrating with respect to $z$

\begin{equation}
    \tilde{\psi}(\omega, x, y, z_0) = \exp\left\{-ik\int_{z=0}^{z=z_0}(\delta_\omega - i
\beta_\omega) \de z \right\} \tilde{\psi}(\omega, x, y, 0),
\label{eq:helmoltz.solution}
\end{equation}
where the effects of the sample perturbations can be described as the sum of
a phase shift determined by the real part of the refractive index $\delta$
\begin{equation*}
    \Delta\varphi(x, y) = -k \int \delta_\omega(x, y, z) \de z
    %\label{eq:phase_deviation}
\end{equation*}
and an attenuation of the intensity, that is the square modulus of the
amplitude $I = |\psi|^2$, according to the Beer-Lambert law
\begin{equation}
    I(\omega, x, y, z_0) = \exp\left\{-2k\int \beta_\omega(x, y, z) \de
    z\right\}
    I(\omega, x, y, 0).\label{eq:beer-lambert}
\end{equation}

The coefficients $\delta$ and $\beta$ can be directly linked to the
interaction of X-rays with the electron clouds of the atoms that constitute
the sample

\begin{equation}
    n = 1 - \delta + i\beta = 1 - \frac{r_e\lambda^2}{2\pi}(f_1 + if_2).
    \label{eq:atom.factors}
\end{equation}

For all comparisons between measured and expected theoretical values in this
thesis, the
tabulated values of $f_1$ and $f_2$ from [NIST citation] together with
equation~\eqref{eq:atom.factors} have been used.

Far from absorption edges, these coefficients depend on the atomic number
$Z$ roughly as $\beta \propto Z^3 / k^4$ and $\delta \propto Z / k^2$,
which leads to the conclusion that the phase term $\delta$ becomes
relatively more important as the energy of the beam is increased. This is
the fundamental motivation of this experimental work, to assess the
feasibility and characteristics of X-ray interferometric techniques for
standard laboratory sources with a higher beam energy.

\section{Interferometry}

The results of the previous sections, and in particular,
equation~\eqref{eq:helmoltz.solution} point to the possibility of recovering
additional information on a sample through the knowledge of the real part of
the refractive index $\delta$. Unfortunately, only the intensity $I =
|\psi|^2$ of the X-ray beam can be directly detected, and an interferometric
setup is needed to convert phase differences into intensity modulations.
Various approaches have been presented in the last decades, on different
applications and source types.

\subsection{Propagation-based phase contrast}
Techniques based on free-space propagation of X-rays have been widely
explored on both monochromatic and polychromatic sources. If a weakly
absorbing object si placed at a distance $R_1$ from a coherent X-ray source,
it will introduce phase differences whose laplacian can be observed as an
intensity modulation at a distance $R_2$ downstream:
\begin{equation}
    I(x, y) = \frac{I_0}{M^2}\left(1 + \frac{R_2\lambda}{2\pi M^2}\nabla_\perp^2
    \varphi(x, y)\right).
    \label{eq:free.space.propagation}
\end{equation}
In general, this technique requires recording such diffraction patterns at
multiple distances $R_2$ from the sample in order to unambiguously
reconstruct the phase and therefore the electron density of the sample.
Moreover, a high spatial coherence of the beam is required, which means
that the source must be very small or placed at a large distance $R_1$ from
the sample. The detector also needs to have a high spatial resolution in
order to resolve the intensity modulations of
equation~\eqref{eq:free.space.propagation}. All of these constraints tend to
have a negative impact on exposure times, but a number of applications are
possible, including tomographic imaging and even \emph{in vivo} imaging.

\subsection{Diffraction enhanced imaging}
Diffraction enhanced --- or crystal-based --- methods employ two crystals:
one after the source, to obtain a monochromatic and collimated X-ray beam;
one before the detector, acting as an extremely sensitive angular filter.

The intensity on the detector is modulated by changing the incidence angle
$\vartheta$ of the beam on the second grating. The intensity in each pixel
is then recorded through multiple exposures as a function of $\vartheta$,
producing a \emph{rocking curve}. These rocking curves are sensitive to
refraction in a direction perpendicular to the beam and to the axis of
rotation of the analyzer grating, and can provide a differential phase
image, as well as a scattering contrast. This technique is also suitable for
polychromatic sources, although the presence of the first crystal acting as
a monochromator with low efficiency also leads to long exposure times,
especially on low brilliance laboratory sources. The beam is also collimated
to a fan beam on the plane and the sample needs to be scanned through it, in
addition to the scanning of the rocking curve. An device composed by an
array of crystals has been proposed as a solution to this issue.

\subsection{Bonse-Hart interferometry}
This method is the oldest approach to X-ray interferometry, where three
silicon beam splitters are used to create two paths from the X-ray source, a
sample is introduced in one of the paths, a second mirror deflects them onto
the third crystal, where they merge again forming an interference pattern
whose characteristics depend on the phase displacements introduced by the
sample. This third crystal is necessary since the fringes of this pattern
have the same period as Bragg planes and are therefore too small to be
directly recorded.

Again, as for diffraction enhanced techniques, the use of crystals as
optical elements means that only a very narrow bandwidth can be used from a
polychromatic source. Moreover, the three beam splitters should be
manufactured from a single crystal in order to reduce instabilities,
resulting in a limitation on the size of the field of view.

\subsection{Edge illumination}
Edge illumination is not an interferometric technique, although it is
inspired by diffraction enhanced imaging. The basic principle of edge
illumination is the collimation of the X-rays through a narrow aperture,
the resulting beam is then masked by a second aperture whose position is
scanned in a direction perpendicular to the beam: the intensity decreases from a
maximum when the two slits are perfectly aligned as the lateral displacement
of the second mask is increased.
The presence of a sample in the beam introduces refraction, which can be
observed as a displacement of this curve. The same applies to each pixel in
the detector, therefore the period of the source and detector masks have to
be matched so that there is a one-to-one mapping from each aperture in the
source mask to each aperture in the detector mask to each underlying
detector pixel.

This approach does not impose requirements on the spatial or temporal
coherence of the source, and has been successfully run on conventional
laboratory sources.

\subsection{Talbot interferometry}
