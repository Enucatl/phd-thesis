%********************************************************************
% Appendix
%*******************************************************
% If problems with the headers: get headings in appendix etc. right
%\markboth{\spacedlowsmallcaps{Appendix}}{\spacedlowsmallcaps{Appendix}}
\chapter{Image reconstruction with TensorFlow\texttrademark}
This appendix reports the technical structure and provides a manual for the
use of the code reconstructing the imaging signals of the grating
interferometer. The calculations are run with the TensorFlow
framework~\cite{tensorflow2015-whitepaper}.
TensorFlow, the TensorFlow logo and any related marks are trademarks of
Google Inc.

The documentation refers to version \texttt{07afdb}, published on 
\href{https://github.com/Enucatl/dpc_reconstruction/tree/420ed0e17fa827c710df1d25b96dc02a45ff0488}{GitHub}~\cite{dpc_reconstruction}.

\section{Goals}
This code takes the raw data with the phase stepping curves from the
detector as input (see section~\ref{sec:gi-image-analysis}) and produces the
three characteristic signals of grating interferometry as output:
transmission, differential phase and dark field.
Both inputs and outputs use the HDF5 file format~\cite{hdf5}.

\section{Prerequisites}
The following libraries and headers should be installed:
libpng, freetype, readline, bzip2, sqlite and hdf5. On Scientific Linux 6 this can be achieved with
\begin{lstlisting}[language=bash]
    su -c 'yum install {libpng,freetype,readline,bzip2,sqlite,hdf5}-devel'
\end{lstlisting}
Additionally, these python packages are required, tested on Python
3.6.4~\cite{python}
\begin{lstlisting}
h5py >= 2.7.1
numpy >= 1.14.0
click >= 6.7
tensorflow >= 1.6
\end{lstlisting}

\section{Installation}
A custom extension is added in C++~\cite{Stroustrup:2013:CPL:2543987} to
retrieve the differential phase signal. After the prerequisites are
installed, the complete package can be compiled and installed simply with

\begin{lstlisting}
make
make install
\end{lstlisting}

\section{Radiography reconstruction pipeline}
The images are read from HDF5 files and stacked into a 3D
\texttt{tensorflow.Tensor} with shape $(x, y, z)$, where $(x, y)$
represents one raw detector image, and the $z$ axis is used for different
phase steps. This dataset is first built for the sample exposure, and one
with the same shape is then constructed for the flat exposure.

After stacking, both \texttt{Tensors} are analyzed with the procedure of
equation~\eqref{eq:flat}: the fast Fourier transform for real inputs is
calculated with \texttt{numpy.fft.rfft} along the $z$ axis, resulting in a
complex output, named \texttt{transformed}. If the grating \G2 was scanned
for a length of $n$ periods of the interference pattern:

\begin{enumerate}
    \item the absolute value of the zeroth component along the $z$ axis is the average value $a_0$
        in equation~\eqref{eq:flat}:
        \texttt{tensorflow.abs(transformed[..., 0])};
    \item the argument of the complex number \texttt{transformed[..., n]}
        is the phase displacement $\theta$ in equation~\eqref{eq:flat}. This
        is calculated with the C++ function \texttt{std::arg}.
    \item the absolute value of the $n$-th component along the $z$ axis is
        the amplitude $a_1$
        in equation~\eqref{eq:flat}:
        \texttt{tensorflow.abs(transformed[..., n])};
\end{enumerate}
The three parameters are then stacked together to form a tensor with shape
$(x, y, 3)$, with $a_0$, $\theta$ and $a_1$ in this order along the third axis.

This calculation is performed on both the sample and flat tensors, and they
are finally compared in the \texttt{compare\_sample\_to\_flat} function to
produce the final output containing a three dimensional
\texttt{tensorflow.Tensor} with shape $(x, y, 3)$, where the indices along
the last axis represent the transmission, differential phase and dark field
image. The code itself is pretty clear in calculating the signals of
section~\ref{sec:gi-image-analysis}

\begin{lstlisting}[language=Python]
import tensorflow as tf

def compare_sample_to_flat(sample, flat):
    a0_flat = flat[..., 0]
    phi_flat = flat[..., 1]
    a1_flat = flat[..., 2]
    result0 = sample[..., 0] / flat[..., 0]
    result1 = tf.mod(
        sample[..., 1] - flat[..., 1] + np.pi,
        2 * np.pi
    ) - np.pi
    result2 = sample[..., 2] / flat[..., 2] / result0
    return tf.stack([result0, result1, result2], axis=-1)
\end{lstlisting}

\section{Command line program}
The pipeline described above can be used on HDF5 files by calling the
\texttt{dpc\_radiography} program. This takes a sequence of HDF5 files, each
file containing \emph{one} phase stepping sequence.
The program runs as an example with
\begin{lstlisting}
dpc_radiography --flats_every 2 --n_flats 3 --group raw_data sample1.h5 sample2.h5 flat1.h5 flat2.h5 flat3.h5 sample3.h5 sample4.h5... flat_last.h5
\end{lstlisting}
if three flat exposures are taken every two sample exposures. These are then
averaged and stacked together in order to form a 3D tensor with the same
shape as the stack of flat images.
The \verb|--group| flag identifies the \texttt{h5py.Group} inside 
each HDF5 file where the datasets are stored. In the example the
files would have the structure:
\begin{lstlisting}
sample1.h5
    | raw_data
        | dataset_phase_step_1
        | dataset_phase_step_2
        | dataset_phase_step_3
        | dataset_phase_step_4
        ...
        | dataset_phase_step_n
\end{lstlisting}

The \verb|--drop_last| flag indicates that the last phase step should be
omitted when taking the Fourier transform, as it is the same as the first.
This happens for instance when scanning across one full grating period and
recording both the beginning and the end of the movement.

The \verb|--overwrite| flag indicates that the output file should be
overwritten if it already exists.

The \verb|--flats_every| flag indicates how many sample files are between
each flat scan. Defaults to 1.

The \verb|--n_flats| flag indicates how many flat scans are to be
averaged every time a flat scan occurs. Defaults to 1.


The \verb|-v| flag can be repeated to increase the verbosity level of the
program, printing more output, useful for debugging. The default behaviour
is to not print any output.

The \verb|--help| flag prints a description of the available command line
flags.

\section{Output structure}
The calculated output saves a new file by taking the first and last file
names together, separated by an underscore. In the example \verb|sample1_flat_last.h5|.
This file contains several datasets, saved inside the \texttt{postprocessing}
group. For a detector with $(x, y)$ pixels, and $n$ phase steps:
\begin{lstlisting}
sample1_flat_last.h5
    | postprocessing
        | dpc_reconstruction         Dataset {x, y, 3}
        | flat_parameters            Dataset {x, y, 3}
        | flat_phase_stepping_curves Dataset {x, y, n}
        | phase_stepping_curves      Dataset {x, y, n}
        | visibility                 Dataset {x, y}
\end{lstlisting}

\begin{description}
    \item[dpc\_reconstruction] contains the transmission, differential
        phase and dark field images, stacked along the last axis;
    \item[flat\_parameters] contains the parameters $a_0$, $\theta$
        and $|a_1|$~\eqref{eq:flat} of the phase stepping curves in the flat
        image;
    \item[flat\_phase\_stepping\_curves] containst the phase stepping
        curves of the flat scans;
    \item[phase\_stepping\_curves] containst the phase stepping
        curves of the sample scans;
    \item[visibility] containst a visibility map, as defined in
        equation~\eqref{eq:visibility-definition}.
\end{description}
