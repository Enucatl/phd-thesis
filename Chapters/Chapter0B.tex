%********************************************************************
% Appendix
%*******************************************************
% If problems with the headers: get headings in appendix etc. right
%\markboth{\spacedlowsmallcaps{Appendix}}{\spacedlowsmallcaps{Appendix}}
\chapter{Laboratory controls with pyepics and HDF5}
This appendix describes the programs used to control the motors and
detectors used throughout the different experiments.

The documentation refers to version \texttt{495333}, published on 
\href{https://github.com/Enucatl/python-controls-high-energy/tree/495333d062067ada0331b8c688dd4067832981fc}{GitHub}~\cite{python-controls-high-energy}.

\section{Prerequisites}
The Eiger and Pilatus detector from Dectris Ltd.\ have been used, which
require the \textsc{albula} software~\cite{albula}. For other detectors and
for the motor controls other packages are required (Python 3.6.4):
\begin{lstlisting}
requests >= 2.18.4
h5py >= 2.7.1
click >= 6.7
numpy >= 1.14.0
scipy >= 1.0.0
decorator >= 4.2.1
pyepics >= 3.3.1
ipython >= 6.2.1
\end{lstlisting}

\section{Installation}
The code is installed on the system's python distribution as a standard
package
\begin{lstlisting}
python setup.py develop
\end{lstlisting}

\section{Motors}
The \texttt{controls.motors.Motor} class controls an EPICS motor, through
its Python interface pyepics. Two methods should be available to the user:
\begin{itemize}
    \item \verb|motor.mv(absolute_position, timeout=9999)| moves the motor
        to the specified absolute position;
    \item \verb|motor.mvr(relative_position, timeout=9999)| moves the motor
        to the specified position, relative to the current position.
\end{itemize}

\section{Detectors}
A generic detector implements the following methods to control the operation
of the detector during scans synchronized with motor movements:
\begin{itemize}
    \item \verb|detector.__init__(host, port, photon_energy, storage_path)|
        connect to the detector at the network address \texttt{host:port}.
        Set an energy threshold \texttt{photon\_energy} (\si{\eV}) and set the
        output directory for the images to \texttt{storage\_path};
    \item \verb|detector.arm()| prepares the detector before an exposure;
    \item \verb|detector.disarm()| hook for post-exposure routines;
    \item \verb|detector.trigger(exposure_time=1)| send a trigger to the
        detector for an exposure of \texttt{exposure\_time} seconds;
    \item \verb|detector.save()| grab all the exposures temporarily stored
        on the detector memory after each \texttt{trigger} call, and save
        them to local destination folder \texttt{storage\_path};
    \item \verb|detector.snap(exposure_time=1)| trigger and save one
        exposure. It calls \texttt{arm}, \texttt{trigger}, \texttt{disarm} and
        \texttt{save}.
\end{itemize}

Calling \texttt{detector.save()} produces a timestamped HDF5 output file inside the
folder \texttt{storage\_path}. The HDF5 file contains all the images
recorded so far, in chronological order, as separate datasets.

\section{Command line operation}
The IPython~\cite{PER-GRA:2007} package allows for the easy programming of
an enhanced Python interpreter, with the forementioned packages. This
command line is started with the \texttt{bunker4controls} command, with
the following arguments:

\begin{itemize}
    \item the \verb|--storage_path| is used to set the output path for all the images
of the session;
    \item the \verb|--threshold| sets the energy threshold of the detector;
    \item the \verb|-v| flag can be repeated to increase the verbosity
        level;
    \item the \verb|--help| flag prints a description of the available command line
        flags.
\end{itemize}

This command then launches the \texttt{controls/scripts/cli.py} program,
initializing all the motors and detectors

\begin{lstlisting}[language=Python]
import click
import IPython
import logging.config
import logging

import controls.motors
import controls.eiger
import controls.remote_detector
import controls.pilatus
import controls.hamamatsu_flat_panel
import controls.comet_tube
import controls.scans
import controls.log_config

@click.command()
@click.option("-v", "--verbose", count=True)
@click.option("-s", "--storage_path",
    default="/afs/psi.ch/project/hedpc/raw_data/2016/pilatus/2016.07.19",
    type=click.Path(exists=True))
@click.option("-t", "--threshold",
    default=10000,
    help="detector threshold energy (eV)")
def main(verbose, storage_path, threshold):
    logger = logging.getLogger()
    logging.config.dictConfig(controls.log_config.get_dict(verbose))
    g0trx = controls.motors.Motor("X02DA-BNK-HE:G0_TRX", "g0trx")
    g0try = controls.motors.Motor("X02DA-BNK-HE:G0_TRY", "g0try")
    g0trz = controls.motors.Motor("X02DA-BNK-HE:G0_TRZ", "g0trz")
    ...
    stptrx = controls.motors.Motor("X02DA-BNK-HE:STP_TRX", "stptrx")
    detector = controls.eiger.Eiger(
    "129.129.99.112",
    storage_path=storage_path,
    photon_energy=threshold)
    IPython.embed()
\end{lstlisting}

This launches an enhanced IPython command line that can be used to move all
the defined motors and detectors. These definitions should be changed in
\texttt{controls/scripts/cli.py} if necessary.
