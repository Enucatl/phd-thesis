%*******************************************************
% Abstract
%*******************************************************
%\renewcommand{\abstractname}{Abstract}
\pdfbookmark[1]{Abstract}{Abstract}
\begingroup
\let\clearpage\relax
\let\cleardoublepage\relax
\let\cleardoublepage\relax

\chapter*{Abstract}
Grating interferometry has been developed on a wide variety of sources and
applications in the last decade, and it is reaching clinical significance in
specific fields such as mammography and lung imaging. This is the
result of the sensitivity of this technique to additional physical signals,
namely differential phase contrast and dark-field contrast, with respect to
conventional transmission radiography. While these
provide great insights in the physical properties of samples, and most
notably soft tissues, thanks to benefits in contrast and sensitivity to
high-frequency components, general purpose applications to medicine or
material sciences have been limited.

One of the main issues is the need for higher photon energy:
between~\num{60} and~\SI{120}{\kilo\volt} for medical imaging, and above for
industrial nondestructive testing. In this work we present the realization
of several interferometers that can be used in this energy range. Moreover
imaging and quantitative modelling are presented, aiming at establishing a
direct link between the recorded images and physical properties of interest
in a sample.

The main results are the realization and study of grating interferometers for
high energies in the edge-on geometry, that is a one-dimensional system
easily adapted to arbitrarily high beam energies; the design and
installation of
two-dimensional systems for laboratory sources with an energy up
to~\SI{120}{\kilo\voltpeak} within the current limitations of grating
fabrication technologies. Finally, experiments on these setups allowed us
to build a complete quantitative model of lung alveolar
microstructures connecting the ground truth as recorded in a high-resolution
microtomographic scan with a synchrotron X-ray source to radiographic
dark-field imaging on a laboratory source.

\newpage

\begin{otherlanguage}{italian}
\pdfbookmark[1]{Sommario}{Sommario}
\chapter*{Sommario}
L'interferometria con reticoli \`e stata sviluppata su una grande variet\`a
di sorgenti e applicazioni nell'ultimo decennio, e sta raggiungendo
risultati di rilevanza clinica in campi specifici, come la mammografia e la
diagnosi di malattie polmonari. Ci\`o risulta dalla sensibilit\`a di questa
tecnica a segnali complementari rispetto alla radiografia di trasmissione,
noti come contrasto differenziale di fase e contrasto di campo oscuro. Per
quanto questi tipi di contrasto siano in
grado di rilevare informazioni sulle propriet\`a fisiche dei campioni, e in
particolare di tessuti molli, grazie a una migliore intensit\`a del
contrasto e una sensibilit\`a a componenti di alta frequenza, le
applicazioni alla medicina e alla scienza dei materiali sono limitate.

Uno dei problemi pi\`u importanti \`e la necessit\`a di utilizzare 
fotoni con energia pi\`u elevata: tra \SI{60}{\kilo\eV} e
\SI{120}{\kilo\eV} per immagini mediche, e oltre per investigazioni
industriali. In questo lavoro presentiamo la realizzazione di vari
interferometri che possono essere utilizzati a queste energie. Inoltre,
si presentano risultati e modelli quantitativi che mirano a stabilire un
collegamento diretto tra le immagini registrate e propriet\`a fisiche del
campione.

I risultati principali sono la realizzazione e lo studio di interferometri
con reticoli per alte energie con geometria di illuminazione laterale,
ovvero un sistema unidimensionale facilmente adattabile a energie
arbitrariamente alte; la progettazione e costruzione di sistemi
bidimensionali per sorgenti da laboratorio con un'energia fino a~\SI{120}{\kilo\voltpeak},
nei limiti delle tecnologie di fabbricazione dei
reticoli. Infine, esperimenti condotti su questi sistemi permettono di
costruire un modello quantitativo completo della struttura alveolare dei
polmoni, che collega la struttura come registrata attraverso microtomografie
ad alta risoluzione da un sincrotrone con radiografie di campo oscuro
registrate con sistemi su sorgenti da laboratorio.
\end{otherlanguage}

\endgroup

\vfill
