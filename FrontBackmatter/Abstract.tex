%*******************************************************
% Abstract
%*******************************************************
%\renewcommand{\abstractname}{Abstract}
\pdfbookmark[1]{Abstract}{Abstract}
\begingroup
\let\clearpage\relax
\let\cleardoublepage\relax
\let\cleardoublepage\relax

\chapter*{Abstract}
Grating interferometry has been developed on a wide variety of sources and
applications in the last decade, and it is reaching clinical significance in
specific fields such as mammography and cartilage imaging. This is the
result of the sensitivity of this technique to additional physical signals,
namely differential phase contrast and dark-field contrast. While these
provide great insights in the physical properties of samples, and most
notably soft tissues, thanks to benefits in contrast and sensitivity to
high-frequency components, general purpose applications to medicine or
material sciences have been limited.

One of the main issues is the need for higher radiation energy:
between~\num{60} and~\SI{120}{\kilo\volt} for medical imaging, and above for
industrial nondestructive testing. In this work we present the realization
of several interferometers that can be used in this energy range. Moreover
imaging and quantitative modelling are presented, aiming at establishing a
direct link between the recorded images and physical properties of interest
in a sample.

The main results are the realization and study of grating interferometers for
high energies in the edge-on geometry, that is a one-dimensional system
easily adapted to arbitrarily high beam energies; the design and
installation of
two-dimensional systems for laboratory sources with an energy up
to~\SI{120}{\kilo\voltpeak} within the current limitations of grating
fabrication technologies. Finally, experiments on these setups allowed us
to build a complete quantitative model of lung alveolar
microstructures connecting the ground truth as recorded in a high-resolution
microtomographic scan with a synchrotron X-ray source to radiographic
dark-field imaging on a laboratory source.

\newpage

\begin{otherlanguage}{italian}
\pdfbookmark[1]{Sommario}{Sommario}
\chapter*{Sommario}
\dots
\end{otherlanguage}

\endgroup

\vfill
